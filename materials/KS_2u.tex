\documentclass{jsarticle}
\usepackage{amsmath, amssymb,amsthm}

\theoremstyle{definition}
\newtheorem*{thm}{定理}

\title{計算数学I\quad \TeX 初級者向けの写経課題}
\begin{document}
\maketitle

本文章は,高木貞治『代数的整数論』の「第1章 代数的整数」の一部を
引用,改変したものである.

\bigskip

代数的の数とは,有理数を係数とする代数方程式の根をいう.
その方程式が一次ならば,根は有理数である.
故に有理数は代数的な数の特別の場合である.
有理係数の方程式を,しばしば,次のような標準的の形に書いて取扱う.

\begin{enumerate}
\renewcommand{\labelenumi}{(\arabic{enumi}${}^\circ$)}
\item 首項の係数で割れば,他の係数は有理数で,方程式の形は
  \[
    x^n+a_1x^{n-1}+\cdots+a_n=0,\qquad \text{$a_i$は有理数.}
  \]
\item 係数の分母を払って,公約数を取り去れば,方程式は
  \[
    a_0x^n+a_1x^{n-1}+\cdots+a_n=0.
  \]
  左辺は所謂原始的多項式である.
  即ち$a_0$,~$a_1$, $\ldots\,$,~$a_n$は整数で,それらの最大公約数は1.
  なお$a_0>0$と仮定してもよい.
\end{enumerate}

\begin{thm}
代数的の数の和は,再び代数的の数となる.
\end{thm}

\begin{proof}
二つの代数的の数$\alpha$,~$\beta$に対し,
$\xi=\alpha+\beta$が代数的の数となることを示す.
$\alpha$,~$\beta$は代数的の数であるから,それぞれを根にもつ有理係数方程式
\begin{align}
  x^m+a_1x^{m-1}+\cdots+a_m&=0,\label{alpha}\\
  x^n+b_1x^{n-1}+\cdots+b_n&=0 \label{beta}
\end{align}
が存在する.

次に,$\mu=0$,~1, $\ldots\,$,~$m-1$と$\nu=0$,~1, $\ldots\,$,~$n-1$に対して,
\begin{equation}
  \alpha^{\mu}\beta^{\nu}\label{multi}
\end{equation}
を辞書式順序で並べたものを
$\omega_1$,~$\omega_2$, $\ldots\,$,~$\omega_l$\ \ ($l=mn$)%
と書けば,各$i=1$,~2, $\ldots\,$,~$l$に対して,対応する$\mu$と$\nu$を用いて
\[
  \xi\omega_i=(\alpha+\beta)\alpha^{\mu}\beta^{\nu}
  =\alpha^{\mu+1}\beta^{\nu}+\alpha^{\mu}\beta^{\nu+1}
\]
が成り立つ.

$0\leq\mu\leq m-2$かつ$0\leq\nu\leq n-2$ならば,
$\xi\omega_i$は\eqref{multi}の二つの数の和に等しい.
また,もし$i=m-1$または$j=n-1$ならば,\eqref{alpha},~\eqref{beta}より
\begin{align*}
  \alpha^m+a_1\alpha^{m-1}+\cdots+a_m&=0,\\
  \beta^n+b_1\beta^{n-1}+\cdots+b_0&=0
\end{align*}
であるから,$\alpha^m=-a_1\alpha^{m-1}-\cdots-a_m$と
$\beta^n=-b_1\beta^{n-1}-\cdots-b_0$を代入することによって,
$\xi\omega_i$を$\omega_1$,~$\omega_2$, $\ldots\,$,~$\omega_l$たちの
有理係数の線型結合として表すことができる.

以上をまとめると,$i=1$,~2, $\ldots\,$,~$l$と$j=1$,~2, $\ldots\,$,~$l$に対して
有理数$c_{ij}$が存在して
\[
  \xi\omega_i=\sum_{j=1}^lc_{ij}\omega_j
\]
が成り立つ.これを用いて,$l$次正方行列の行列式に関する方程式
\[
  \begin{vmatrix}
    c_{11}-\xi &c_{12}     &\cdots &c_{1l}\\
    c_{21}     &c_{22}-\xi &\cdots &c_{2l}\\
    \vdots     &\vdots     &\ddots &\vdots\\
    c_{l1}     &c_{l2}     &\cdots &c_{ll}-\xi
  \end{vmatrix}
  =0
\]
を考える.左辺を展開すれば,
\[
  \xi^l+{\tilde{c}}_1\xi^{l-1}+\cdots+{\tilde{c}}_l=0
\]
を得る.
ここに,各${\tilde{c}}_i$は有理数$c_{ij}$たちの整式として表される実数である.
特に,各${\tilde{c}}_i$も有理数であるから,$\xi$は代数的の数である.
\end{proof}
\end{document}